\documentclass{report}

% Packages for math symbols and equations

\usepackage[
natbib,
style=alphabetic,
maxbibnames=10,  
sorting=ydnt,
url=false,
doi=false,
sortcites,
defernumbers,
backref,
backend=biber
]{biblatex}
\addbibresource{bibliography.bib}
\usepackage{soul} % For strikethrough

% Title page information
\title{Grammaire française}
\author{Giorgos}
\date{\today}

\begin{document}

\maketitle

\tableofcontents

\chapter{Subjonctif présent}
\section{Formation}
Le subjonctif présent est formé sur la racine de la \textbf{3ème personne} du pluriel de l'indicatif présent. On ajoute les terminaisons suivantes:
\begin{center}
    -e, -es, -e, -ions, -iez, -ent    
\end{center}
\section{Emploi}
Pour utiliser le subjonctif présent, il faut que le sujet de la principale soit différent du sujet de la subordonnée. Si c'est pas le cas, on utilise l'indicatif.
Par exemple:
\begin{itemize}
    \item \st{Je veux que je sois heureux} \textrightarrow Je veux être heureux
    \item \st{Tu es triste que tu ne puisses pas venir à la fête} \textrightarrow Tu es triste de ne pas pouvoir venir à la fête
    \item \st{Il doute qu’il réussisse son examen} \textrightarrow Il doute de réussir son examen
\end{itemize}
On utilise le subjonctif présent après les verbes suivants:
\begin{center}
    \begin{tabular}{|p{0.45\textwidth}|p{0.45\textwidth}|}
        \hline
        \textbf{Volonté, Obligation, Nécessité} & \textbf{Sentiments, émotions} \\
        \hline
        \begin{itemize}
            \item vouloir, avoir envie, désirer, souhaiter, accepter, refuser que
            \item être d’accord/ne pas être d’accord que
            \item aimer que
            \item avoir besoin que
            \item exiger, ordonner, demander que
            \item Il faut que
            \item Il est urgent, important, nécessaire, essentiel que
        \end{itemize} 
        & 
        \begin{itemize}
            \item aimer, apprécier, regretter que
            \item avoir peur, craindre que
            \item être triste, désolé, heureux, content que
            \item être, surpris, étonné, fâché, scandalisé que
            \item c’est dommage, génial que
            \item \textbf{Attention:} Espérer que est suivi de l'indicatif.
        \end{itemize} \\
        \hline
        \textbf{Doute} & \textbf{Opinion, forme négative, forme interrogative} \\
        \hline
        \begin{itemize}
            \item douter que
            \item ne pas être sûr que
            \item il est possible/impossible que
        \end{itemize} 
        & 
        \begin{itemize}
            \item ne pas penser / trouver / croire que
            \item Crois-tu que … ?
            \item \textbf{Attention:} Penser / trouver / croire que est suivi de l'indicatif.
        \end{itemize} \\
        \hline
    \end{tabular}
\end{center}
On l'utilise aussi avec les conjonctions suivantes:
\begin{itemize}
    \item (But) pour que, afin que
    \item (Temps) avant que, jusqu’à ce que, en attendant que
    \item (Conséquence) quoique (formel), bien que
    \item (Condition) à condition que
    \item (Manière) sans que
\end{itemize}


\chapter{Relations logiques}
Les relations logiques les plus importantes sont les suivantes:

\begin{center}
    \begin{tabular}{|p{0.45\textwidth}|p{0.45\textwidth}|}
        \hline
        \textbf{Cause} & \textbf{Conséquence} \\
        \hline
        \begin{itemize}
            \item \textbf{à cause de} + nom (cause négative)
            \item \textbf{grâce à} + nom (cause positive)
            \item \textbf{en raison de} + nom (formel)
            \item \textbf{parce que} + indicatif
            \item \textbf{car} + indicatif
            \item \textbf{puisque} + indicatif (cause évidente ou connue)
            \item \textbf{comme} + indicatif (en début de phrase)
        \end{itemize} 
        & 
        \begin{itemize}
            \item \textbf{donc}
            \item \textbf{alors}
            \item \textbf{par conséquent}
            \item \textbf{c'est pourquoi}
            \item \textbf{c'est la raison pour laquelle}
            \item verbe + \textbf{tellement que} ou \textbf{tellement de} + nom
            \item \textbf{si} + adjectif + \textbf{que}
        \end{itemize} \\
        \hline
        \end{tabular}    
\end{center}

\begin{center}
    \textbf{Les principales conjonctions}
    \begin{tabular}{|p{0.3\textwidth}|p{0.3\textwidth}|p{0.3\textwidth}|}
        \hline
        \textbf{+ indicatif} & \textbf{+ subjonctif} & \textbf{+ conditionnel} \\
        \hline
        Contenu 1 & Contenu 2 & Contenu 3 \\
        \hline
        Contenu 4 & Contenu 5 & Contenu 6 \\
        \hline
        Contenu 7 & Contenu 8 & Contenu 9 \\
        \hline
    \end{tabular}
\end{center}

\printbibliography
\end{document}